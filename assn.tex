\documentclass[12pt,a4paper]{article}
\usepackage[a4paper,left=3cm,right=2cm,top=2cm,bottom=2cm]{geometry}

\pagenumbering{gobble}
\title{\LaTeX}

\author{
	\hspace{4in}{Dheeraj Athrey}\\
}

\begin{document}
\maketitle


\section*{Introduction}


The methods used for multiplication in this article are methods found in the ancient Indian veda, Atharva veda\cite{bloomfield1899atharvaveda}\cite{India1000shaping}. These are easier and faster methods compared to the general multiplication method forl numbers, but for big numbers they wouldn't be that faster.\\
The sytax used to generate the below pages can be found inthe website Wiki books\cite{website:Wikibooks}.\\


The Atharvaveda is composed in Vedic Sanskrit, and it is a collection of 730 hymns with about 6,000 mantras, divided into 20 books.\\
About a sixth of the Atharvaveda text adapts verses from the Rigveda, and except for Books 15 and 16, the text is in poem form deploying a diversity of Vedic meters.\\


\bibliography{assn}

\bibliographystyle{plain}



\end{document}